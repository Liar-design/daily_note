% 声明文件类型
\documentclass[12pt,a4paper,oneside]{ctexart}


\CTEXsetup[format={\Large\bfseries}]{section}

% 声明所需要用到的宏包
\usepackage{amsmath}
\usepackage{hyperref}
\usepackage{bookmark}
\usepackage{graphicx}

\usepackage{titlesec}

% 导言区
\title{\vspace{-4cm}\textbf{数字通信课程学习报告}}
\author{徐淳 231607010094}

% 主体框架
\begin{document}
\maketitle
\section{引言}  
在本次课程学习中,我习得了较为全面的数字通信理论知识,建立起了相关的理论框架。本次课程从介绍相关概念,复习概率论相关知识开始讲起,后面又介绍了随机过程、基带与滤波通带、信号空间等内容,本次报告针对课程伊始章节Stachastic Process(随机过程)作相应学习笔记及报告。
\section{随机过程的基本概念}
随机过程是一类随时间作随机变化的过程,它不能用确切的时间函数描述。如果把把随机过程看成对应不同随机试验结果的时间过程的集合,即随机过程是所有样本函数的集合。具体而言即测试结果的每一个记录,即图中的每一个波形,都是一个确定时间函数$X_{i}(t)$,它称为样本函数或随机过程的一次实现。全部样本函数构成的总体就是一个随机过程,记作$\xi(t)$。
    \subsection{相关概念}
        \begin{itemize}
            \item[$\bullet$] 随机事件(简称事件、实现样本、记录):在随机试验中,可能出现也可能不出现,而在大量重复试验中具有某种规律性的事件
            \item[$\bullet$] 随机变量:定义在样本空间上的实值函数,是不确定的
            \item[$\bullet$] 随机过程:无数个随机变量构成的总体
            \item[$\bullet$] 离散随机变量:随机变量x的取值个数是有限的或可数无穷个
            \item[$\bullet$] 连续随机变量:随机变量x可能的取值充满某一有限或无限区间  
        \end{itemize}
    \subsection{随机过程的分布函数}
    由$\xi(t)$表示一个随机过程,在任意一个时刻$t_{1}$上$\xi (t_{1})$是一个随机变量,定义随机过程$\xi(t)$的\\
    一维概率分布函数:
    $$
    F_{1} (x_{1};t_{1}) = P\left \{ \xi(t_{1})\le  x_{1} \right \}
    $$
    一维概率密度函数:
    $$
    f_{1} (x_{1};t_{1}) = \frac{\partial F_{1} (x_{1};t_{1})}{\partial x_{1}}
    $$
    或
    $$F_{1} (x_{1};t_{1}) = \int_{-\infty }^{x}  f_{1} (x_{1};t_{1})dx
    $$
    n维概率分布函数:
    $$
    F_{n} (x_{1},x_{2}...,x_{n};t_{1},t_{2},...,t_{n}) = P\left \{ \xi(t_{1})\le  x_{1} ,\xi(t_{1})\le  x_{1},...,\xi(t_{1})\le  x_{1}\right \}
    $$
    n维概率密度函数:
    $$
    f_{n} (x_{1},x_{2}...,x_{n};t_{1},t_{2},...,t_{n}) = \frac{\partial F_{1} (x_{1},x_{2},...,x_{n};t_{1},t_{2},...,t_{n})}{\partial x_{1}\partial x_{2}...\partial x_{n}}
    $$
    或
    $$
    F_{n} (x_{1},x_{2}...,x_{n};t_{1},t_{2},...,t_{n}) = \int\limits_{-\infty }^{x_{1}} \int\limits_{-\infty }^{x_{2}}...\int\limits_{-\infty }^{x_{n}}f_{n}(x_{1},x_{2},...,x_{n};t_{1},t_{2},...t_{n})dx_{1}dx_{2}...dx_{n}
    $$
\section{随机过程的数字特征}
1.均值(数学期望)\\
\indent 随机过程$\xi(t)$的数学期望:
    \begin{itemize}
        \item [$\bullet$] 若$\xi(t)$连续,定义$E[\xi(t)] = \int_{-\infty }^{\infty } xf_{1}(x,t)dx = a(t)$
        \item [$\bullet$] 若$\xi(t)$离散,定义$E[\xi(t)] =  {\textstyle \sum_{i=1}^{K}}\xi_{i}(t)P(\xi_{i}(t))=a(t)$ 
    \end{itemize}
\indent \indent $\xi (t)$的均值$E\left [ \xi(t) \right ] $是时间的确定函数,常记为$a(t)$,他表示随机过程的$n$个样本函数曲线的摆动中心\\

\indent 2.方差\\
\indent 随机过程$\xi(t)$的方差:
$$
D\left[\xi(t)\right] = E\left\{\left[\xi(t)-a(t)\right]^{2}\right\} = E[\xi^{2}(t)] - a^{2}(t) = \sigma^{2}(t)
$$
\indent 方差等于均方值与均方平方之差,表示随机过程在时刻$t$相对于均值$a(t)$的偏离程度

\indent 3.相关函数\\
\indent $\xi(t)$的协方差函数:
$$
B[t_{1},t_{2}] = E\left\{[\xi(t_{1}) - a(t1)][\xi(t_{2}) - a(t_{2})]\right\} 
$$
$$
= \int_{-\infty }^{\infty } \int_{-\infty}^{\infty}[x_{1} - a(t)][x_{2} - a(t_{2})]f_{2}(x_{1},x_{2};t_{1},t_{2})dx_{1}dx_{2}
$$
\indent 式中:$a(t_{1})$和$a(t_{2})$分别是在$t_{1}$和$t_{2}$时刻得到的$\xi(t)$的均值;$f_{2}(x_{1},x_{2};t_{1},t_{2})$为$\xi(t)$的二维度概率密度函数。

\indent $\xi(t)$的相关函数:
$$
R[t_{1},t_{2}] = E[\xi(t_{1})\xi(t_{2})] = \int_{-\infty}^{\infty}\int_{-\infty}^{\infty}x_{1}x_{2}f_{2}(x_{1},x_{2};t_{1},t_{2})dx_{1}dx_{2}
$$
\indent 式中:$\xi(t_{1})$和$\xi(t_{1})$分别是在$t_{1}$和$t_{2}$时刻观测$\xi(t)$得到的随机变量
\section{关于平稳随机过程}


\end{document}